%% Generated by Sphinx.
\def\sphinxdocclass{report}
\documentclass[a4paper,10pt,icelandic]{sphinxmanual}
\ifdefined\pdfpxdimen
   \let\sphinxpxdimen\pdfpxdimen\else\newdimen\sphinxpxdimen
\fi \sphinxpxdimen=.75bp\relax
\ifdefined\pdfimageresolution
    \pdfimageresolution= \numexpr \dimexpr1in\relax/\sphinxpxdimen\relax
\fi
%% let collapsible pdf bookmarks panel have high depth per default
\PassOptionsToPackage{bookmarksdepth=5}{hyperref}
%% turn off hyperref patch of \index as sphinx.xdy xindy module takes care of
%% suitable \hyperpage mark-up, working around hyperref-xindy incompatibility
\PassOptionsToPackage{hyperindex=false}{hyperref}
%% memoir class requires extra handling
\makeatletter\@ifclassloaded{memoir}
{\ifdefined\memhyperindexfalse\memhyperindexfalse\fi}{}\makeatother

\PassOptionsToPackage{booktabs}{sphinx}
\PassOptionsToPackage{colorrows}{sphinx}

\PassOptionsToPackage{warn}{textcomp}
\usepackage[utf8]{inputenc}
\ifdefined\DeclareUnicodeCharacter
% support both utf8 and utf8x syntaxes
  \ifdefined\DeclareUnicodeCharacterAsOptional
    \def\sphinxDUC#1{\DeclareUnicodeCharacter{"#1}}
  \else
    \let\sphinxDUC\DeclareUnicodeCharacter
  \fi
  \sphinxDUC{00A0}{\nobreakspace}
  \sphinxDUC{2500}{\sphinxunichar{2500}}
  \sphinxDUC{2502}{\sphinxunichar{2502}}
  \sphinxDUC{2514}{\sphinxunichar{2514}}
  \sphinxDUC{251C}{\sphinxunichar{251C}}
  \sphinxDUC{2572}{\textbackslash}
\fi
\usepackage{cmap}
\usepackage[T1]{fontenc}
\usepackage{amsmath,amssymb,amstext}
\usepackage[icelandic]{babel}


\usepackage{lmodern}
\usepackage[T1]{fontenc}


\usepackage[Sonny]{fncychap}
\ChNameVar{\Large\normalfont\sffamily}
\ChTitleVar{\Large\normalfont\sffamily}
\usepackage[,numfigreset=1,mathnumfig]{sphinx}

\fvset{fontsize=auto}
\usepackage{geometry}

\usepackage{sphinxcontribtikz}

% Include hyperref last.
\usepackage{hyperref}
% Fix anchor placement for figures with captions.
\usepackage{hypcap}% it must be loaded after hyperref.
% Set up styles of URL: it should be placed after hyperref.
\urlstyle{same}


\usepackage{sphinxmessages}
\setcounter{tocdepth}{5}
\setcounter{secnumdepth}{5}


    % Koma í veg fyrir að kaflar byrji á oddatölusíðum
        \let\cleardoublepage\clearpage

    % Sérsniðin skipun til að meðhöndla línubil í höfunda hlutanum á forsíðu
        \newcommand{\newlineauthors}[1]{\parbox{0.8\textwidth}{\raggedleft#1}}

    % Fjarlægja dagsetningu af forsíðu
        \AtBeginDocument{\date{}}

    % Skilgreina litinn Blue Nova Deep
        \definecolor{bluenovadeep}{rgb}{0.192,0.255,0.604}

    % Skrá stillingar fyrir "hypersetup"
        \hypersetup{
            bookmarksnumbered=true,     % Kaflanúmer koma fram í Bookmarks
            bookmarksopen=true,         % Bookmarks eru alltaf opin
            bookmarksopenlevel=0,       % Bookmarks eru alltaf opin upp að kafla (en ekki undirkafla)
            pdfnewwindow=true,          % Tenglar opnast í nýjum glugga í vafra (virkar samt ekki í öllum vöfrum)
            colorlinks=true,            % Tenglar birtast með litum
            linkcolor=black,            % "linkcolor" er svartur og inniheldur liti á tenglum í efnisyfirliti
            urlcolor=bluenovadeep,      % "urlcolor" er Blue Nova Deep og inniheldur liti á tenglum á forsíðu og inline tenglum
            citecolor=black,            % "citecolor" er svartur og inniheldur líklega liti á tenglum í genindex, o.fl. (todo)
        }

    % Skrá stillingar fyrir "titlesec" (notað hér til að velja liti á fyrirsögnum)
        \usepackage{titlesec}
        \titleformat{\section}
            {\normalfont\Large\bfseries\color{black}}{\thesection}{1em}{\bfseries}
        \titleformat{\subsection}
            {\normalfont\large\bfseries\color{black}}{\thesubsection}{1em}{}
        \titleformat{\subsubsection}
            {\normalfont\normalsize\bfseries\color{black}}{\thesubsubsection}{1em}{}
        \titleformat{\paragraph}
            {\normalfont\normalsize\bfseries\color{black}}{\theparagraph}{1em}{}
        \titleformat{\subparagraph}
            {\normalfont\normalsize\bfseries\color{black}}{\thesubparagraph}{1em}{}
    

\title{Kitchen sink}
\date{maí 30, 2024}
\release{}
\author{\newlineauthors{\large{~}\\[5em] \Large{Ritsafn RÚBIK Reykjavíkur (\href{https://rit.rubik.is}{rit.rubik.is})}\\[1em]}\\ \newlineauthors{\normalsize{\textmd{\textsf{Eigandi efnis og leyfisveitandi:}}}\\[0cm] \large{\textmd{\textsf{RÚBIK Reykjavík ehf. (\href{mailto:rubik@rubik.is}{rubik@rubik.is})}}}\\[1em] \normalsize{\textmd{\textsf{Höfundur efnis:}}}\\[0cm] \large{\textmd{\textsf{Atli Bjarnason (\href{mailto:rubik@rubik.is}{a@rubik.is})}}}\\[16em]}\\ \newlineauthors{\normalsize{Ritsafn RÚBIK Reykjavíkur © RÚBIK Reykjavík ehf.}\\[0.3em] \normalsize{\textmd{\textsf{Notkun efnis er heimil samkvæmt \href{https://github.com/rubikrvk/ritsafn/blob/main/LICENSE}{notkunarleyfi} Creative Commons Attribution-NonCommercial-ShareAlike 4.0 International (\href{https://creativecommons.org/licenses/by-nc-sa/4.0/deed.is}{CC BY-NC-SA 4.0}).}}}\\[0em]}}
\newcommand{\sphinxlogo}{\sphinxincludegraphics{rubik-cover.png}\par}
\renewcommand{\releasename}{}
\makeindex
\begin{document}

\ifdefined\shorthandoff
  \ifnum\catcode`\=\string=\active\shorthandoff{=}\fi
  \ifnum\catcode`\"=\active\shorthandoff{"}\fi
\fi

\pagestyle{empty}
\sphinxmaketitle
\pagestyle{plain}
\sphinxtableofcontents
\pagestyle{normal}
\phantomsection\label{\detokenize{index::doc}}


\sphinxstepscope


\chapter{Admonitions}
\label{\detokenize{admonitions/index:admonitions}}\label{\detokenize{admonitions/index::doc}}
\sphinxAtStartPar
Text below header one


\section{Header two}
\label{\detokenize{admonitions/index:header-two}}
\sphinxAtStartPar
Text below header two


\subsection{Header three}
\label{\detokenize{admonitions/index:header-three}}
\sphinxAtStartPar
Text below header three


\subsubsection{Header four}
\label{\detokenize{admonitions/index:header-four}}
\sphinxAtStartPar
Text below header four


\paragraph{Header five}
\label{\detokenize{admonitions/index:header-five}}
\sphinxAtStartPar
Text below header five


\subparagraph{Header six}
\label{\detokenize{admonitions/index:header-six}}
\sphinxAtStartPar
Text below header six

\sphinxstepscope


\chapter{Blocks}
\label{\detokenize{blocks/index:blocks}}\label{\detokenize{blocks/index::doc}}
\sphinxAtStartPar
Text below header one


\section{Header two}
\label{\detokenize{blocks/index:header-two}}
\sphinxAtStartPar
Text below header two


\subsection{Header three}
\label{\detokenize{blocks/index:header-three}}
\sphinxAtStartPar
Text below header three


\subsubsection{Header four}
\label{\detokenize{blocks/index:header-four}}
\sphinxAtStartPar
Text below header four


\paragraph{Header five}
\label{\detokenize{blocks/index:header-five}}
\sphinxAtStartPar
Text below header five


\subparagraph{Header six}
\label{\detokenize{blocks/index:header-six}}
\sphinxAtStartPar
Text below header six

\sphinxstepscope


\chapter{Generic items}
\label{\detokenize{generic-items/index:generic-items}}\label{\detokenize{generic-items/index::doc}}
\sphinxAtStartPar
Text below header one


\section{Header two}
\label{\detokenize{generic-items/index:header-two}}
\sphinxAtStartPar
Text below header two


\subsection{Header three}
\label{\detokenize{generic-items/index:header-three}}
\sphinxAtStartPar
Text below header three


\subsubsection{Header four}
\label{\detokenize{generic-items/index:header-four}}
\sphinxAtStartPar
Text below header four


\paragraph{Header five}
\label{\detokenize{generic-items/index:header-five}}
\sphinxAtStartPar
Text below header five


\subparagraph{Header six}
\label{\detokenize{generic-items/index:header-six}}
\sphinxAtStartPar
Text below header six

\sphinxstepscope


\chapter{Lists}
\label{\detokenize{lists/index:lists}}\label{\detokenize{lists/index::doc}}
\sphinxAtStartPar
Text below header one


\section{Header two}
\label{\detokenize{lists/index:header-two}}
\sphinxAtStartPar
Text below header two


\subsection{Header three}
\label{\detokenize{lists/index:header-three}}
\sphinxAtStartPar
Text below header three


\subsubsection{Header four}
\label{\detokenize{lists/index:header-four}}
\sphinxAtStartPar
Text below header four


\paragraph{Header five}
\label{\detokenize{lists/index:header-five}}
\sphinxAtStartPar
Text below header five


\subparagraph{Header six}
\label{\detokenize{lists/index:header-six}}
\sphinxAtStartPar
Text below header six

\sphinxstepscope


\chapter{Structural elements}
\label{\detokenize{structural-elements/index:structural-elements}}\label{\detokenize{structural-elements/index::doc}}
\sphinxAtStartPar
Text below header one


\section{Header two}
\label{\detokenize{structural-elements/index:header-two}}
\sphinxAtStartPar
Text below header two


\subsection{Header three}
\label{\detokenize{structural-elements/index:header-three}}
\sphinxAtStartPar
Text below header three


\subsubsection{Header four}
\label{\detokenize{structural-elements/index:header-four}}
\sphinxAtStartPar
Text below header four


\paragraph{Header five}
\label{\detokenize{structural-elements/index:header-five}}
\sphinxAtStartPar
Text below header five


\subparagraph{Header six}
\label{\detokenize{structural-elements/index:header-six}}
\sphinxAtStartPar
Text below header six

\sphinxstepscope


\chapter{Tables}
\label{\detokenize{tables/index:tables}}\label{\detokenize{tables/index::doc}}
\sphinxAtStartPar
Text below header one


\section{Header two}
\label{\detokenize{tables/index:header-two}}
\sphinxAtStartPar
Text below header two


\subsection{Header three}
\label{\detokenize{tables/index:header-three}}
\sphinxAtStartPar
Text below header three


\subsubsection{Header four}
\label{\detokenize{tables/index:header-four}}
\sphinxAtStartPar
Text below header four


\paragraph{Header five}
\label{\detokenize{tables/index:header-five}}
\sphinxAtStartPar
Text below header five


\subparagraph{Header six}
\label{\detokenize{tables/index:header-six}}
\sphinxAtStartPar
Text below header six

\sphinxstepscope


\chapter{Tikz graphics}
\label{\detokenize{tikz-graphics/index:tikz-graphics}}\label{\detokenize{tikz-graphics/index::doc}}
\sphinxAtStartPar
Text below header one


\section{Header two}
\label{\detokenize{tikz-graphics/index:header-two}}
\sphinxAtStartPar
Text below header two


\subsection{Header three}
\label{\detokenize{tikz-graphics/index:header-three}}
\sphinxAtStartPar
Text below header three


\subsubsection{Header four}
\label{\detokenize{tikz-graphics/index:header-four}}
\sphinxAtStartPar
Text below header four


\paragraph{Header five}
\label{\detokenize{tikz-graphics/index:header-five}}
\sphinxAtStartPar
Text below header five


\subparagraph{Header six}
\label{\detokenize{tikz-graphics/index:header-six}}
\sphinxAtStartPar
Text below header six

\sphinxstepscope


\chapter{MathJax renders}
\label{\detokenize{mathjax-renders/index:mathjax-renders}}\label{\detokenize{mathjax-renders/index::doc}}
\sphinxAtStartPar
Text below header one


\section{Header two}
\label{\detokenize{mathjax-renders/index:header-two}}
\sphinxAtStartPar
Text below header two


\subsection{Header three}
\label{\detokenize{mathjax-renders/index:header-three}}
\sphinxAtStartPar
Text below header three


\subsubsection{Header four}
\label{\detokenize{mathjax-renders/index:header-four}}
\sphinxAtStartPar
Text below header four


\paragraph{Header five}
\label{\detokenize{mathjax-renders/index:header-five}}
\sphinxAtStartPar
Text below header five


\subparagraph{Header six}
\label{\detokenize{mathjax-renders/index:header-six}}
\sphinxAtStartPar
Text below header six



\renewcommand{\indexname}{Yfirlit}
\printindex
\end{document}