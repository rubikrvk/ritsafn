%% Generated by Sphinx.
\def\sphinxdocclass{report}
\documentclass[a4paper,10pt,icelandic]{sphinxmanual}
\ifdefined\pdfpxdimen
   \let\sphinxpxdimen\pdfpxdimen\else\newdimen\sphinxpxdimen
\fi \sphinxpxdimen=.75bp\relax
\ifdefined\pdfimageresolution
    \pdfimageresolution= \numexpr \dimexpr1in\relax/\sphinxpxdimen\relax
\fi
%% let collapsible pdf bookmarks panel have high depth per default
\PassOptionsToPackage{bookmarksdepth=5}{hyperref}

\PassOptionsToPackage{booktabs}{sphinx}
\PassOptionsToPackage{colorrows}{sphinx}

\PassOptionsToPackage{warn}{textcomp}
\usepackage[utf8]{inputenc}
\ifdefined\DeclareUnicodeCharacter
% support both utf8 and utf8x syntaxes
  \ifdefined\DeclareUnicodeCharacterAsOptional
    \def\sphinxDUC#1{\DeclareUnicodeCharacter{"#1}}
  \else
    \let\sphinxDUC\DeclareUnicodeCharacter
  \fi
  \sphinxDUC{00A0}{\nobreakspace}
  \sphinxDUC{2500}{\sphinxunichar{2500}}
  \sphinxDUC{2502}{\sphinxunichar{2502}}
  \sphinxDUC{2514}{\sphinxunichar{2514}}
  \sphinxDUC{251C}{\sphinxunichar{251C}}
  \sphinxDUC{2572}{\textbackslash}
\fi
\usepackage{cmap}
\usepackage[T1]{fontenc}
\usepackage{amsmath,amssymb,amstext}
\usepackage[icelandic]{babel}


\usepackage{lmodern}
\usepackage[T1]{fontenc}


\usepackage[Sonny]{fncychap}
\ChNameVar{\Large\normalfont\sffamily}
\ChTitleVar{\Large\normalfont\sffamily}
\usepackage[,numfigreset=1,mathnumfig]{sphinx}

\fvset{fontsize=auto}
\usepackage{geometry}

\usepackage{sphinxcontribtikz}

% Include hyperref last.
\usepackage{hyperref}
% Fix anchor placement for figures with captions.
\usepackage{hypcap}% it must be loaded after hyperref.
% Set up styles of URL: it should be placed after hyperref.
\urlstyle{same}

\addto\captionsicelandic{\renewcommand{\contentsname}{Eldhúsvaskur}}

\usepackage{sphinxmessages}
\setcounter{tocdepth}{5}
\setcounter{secnumdepth}{5}


    % Koma í veg fyrir að kaflar byrji á oddatölusíðum
        \let\cleardoublepage\clearpage

    % Sérsniðin skipun til að meðhöndla línubil í höfunda hlutanum á forsíðu
        \newcommand{\newlineauthors}[1]{\parbox{0.8\textwidth}{\raggedleft#1}}

    % Skilgreina "currentyear" sem núverandi ár
        \newcommand{\currentyear}{\the\year}

    % Fjarlægja dagsetningu af forsíðu
        \AtBeginDocument{\date{}}

    % Skilgreina litinn Blue Nova Deep
        \definecolor{bluenovadeep}{rgb}{0.192,0.255,0.604}

    % Skrá stillingar fyrir "hypersetup"
        \hypersetup{
            bookmarksnumbered=true,     % Kaflanúmer koma fram í Bookmarks
            bookmarksopen=true,         % Bookmarks eru alltaf opin
            bookmarksopenlevel=0,       % Bookmarks eru alltaf opin upp að kafla (en ekki undirkafla)
            pdfnewwindow=true,          % Tenglar opnast í nýjum glugga í vafra (virkar samt ekki í öllum vöfrum)
            colorlinks=true,            % Tenglar birtast með litum
            linkcolor=black,            % "linkcolor" er svartur og inniheldur liti á tenglum í efnisyfirliti
            urlcolor=bluenovadeep,      % "urlcolor" er Blue Nova Deep og inniheldur liti á tenglum á forsíðu og inline tenglum
            citecolor=black,            % "citecolor" er svartur og inniheldur líklega liti á tilvísunum úr t.d. bibtex, o.fl. (todo)
        }

    % Skrá stillingar fyrir "titlesec" (notað hér til að velja liti á fyrirsögnum)
        \usepackage{titlesec}
        \titleformat{\section}
            {\normalfont\Large\bfseries\color{black}}{\thesection}{1em}{\bfseries}
        \titleformat{\subsection}
            {\normalfont\large\bfseries\color{black}}{\thesubsection}{1em}{}
        \titleformat{\subsubsection}
            {\normalfont\normalsize\bfseries\color{black}}{\thesubsubsection}{1em}{}
        \titleformat{\paragraph}
            {\normalfont\normalsize\bfseries\color{black}}{\theparagraph}{1em}{}
        \titleformat{\subparagraph}
            {\normalfont\normalsize\bfseries\color{black}}{\thesubparagraph}{1em}{}

    % Skrá " endash " í staðinn fyrir ": " á milli numfig_format og caption, og gera numfig_format italic
        \usepackage{caption}
        \captionsetup{labelsep=endash, labelfont={bf}}
    

\title{Eldhúsvaskur}
\date{jún. 10, 2024}
\release{}
\author{\newlineauthors{\large{~}\\[5em] \Large{Ritsafn RÚBIK Reykjavíkur (\href{https://rit.rubik.is}{rit.rubik.is})}\\[1em]}\\ \newlineauthors{\normalsize{\textmd{\textsf{Eigandi efnis og leyfisveitandi:}}}\\[0cm] \large{\textmd{\textsf{RÚBIK Reykjavík ehf. (\href{mailto:rubik@rubik.is}{rubik@rubik.is})}}}\\[1em] \normalsize{\textmd{\textsf{Höfundur efnis:}}}\\[0cm] \large{\textmd{\textsf{Atli Bjarnason (\href{mailto:rubik@rubik.is}{a@rubik.is})}}}\\[16em]}\\ \newlineauthors{\normalsize{Ritsafn RÚBIK Reykjavíkur © 2023--\currentyear\ RÚBIK Reykjavík ehf.}\\[0.3em] \small{\textmd{\textsf{Notkun efnis úr Ritsafni RÚBIK Reykjavíkur er heimil samkvæmt \href{https://github.com/rubikrvk/ritsafn/blob/main/LICENSE}{notkunarleyfi} Creative Commons Attribution-NonCommercial-ShareAlike 4.0 International (\href{https://creativecommons.org/licenses/by-nc-sa/4.0/deed.is}{CC BY-NC-SA 4.0}).}}}\\[0em]}}
\newcommand{\sphinxlogo}{\sphinxincludegraphics{rubik-cover.png}\par}
\renewcommand{\releasename}{}
\makeindex
\begin{document}

\ifdefined\shorthandoff
  \ifnum\catcode`\=\string=\active\shorthandoff{=}\fi
  \ifnum\catcode`\"=\active\shorthandoff{"}\fi
\fi

\pagestyle{empty}
\sphinxmaketitle
\pagestyle{plain}
\sphinxtableofcontents
\pagestyle{normal}
\phantomsection\label{\detokenize{index::doc}}


\sphinxAtStartPar
Og svo vil ég indexa þetta \index{hugtak í eldhúsvaski@\spxentry{hugtak í eldhúsvaski}}hugtak í eldhúsvaski.

\sphinxAtStartPar
Setjum \(6b_y=-6b_x\) inn og fáum:
\begin{equation*}
\begin{split}6b_y = -6b_x\end{split}
\end{equation*}\begin{equation}\label{equation:index:euler5}
\begin{split}e^{i\pi} + 1 = 0\end{split}
\end{equation}
\index{Euler\textquotesingle{}s identity@\spxentry{Euler\textquotesingle{}s identity}}\ignorespaces 
\sphinxAtStartPar
Smá annað test:
\begin{equation}\label{equation:index:euler6}
\begin{split}e^{i\pi} + 2 = 0\end{split}
\end{equation}
\sphinxAtStartPar
Euler’s identity, equation (\ref{equation:index:euler6}), was elected one of the most
beautiful mathematical formulas.

\sphinxAtStartPar
This is a test. Here is an equation:
\(X_{0:5} = (X_0, X_1, X_2, X_3, X_4)\).
Here is another:

\begin{figure}[htbp]
\centering
\capstart

\noindent\sphinxincludegraphics[width=200\sphinxpxdimen,height=200\sphinxpxdimen]{{pexels-photo-106399}.jpeg}
\caption{Þetta er caption fyrir litla mynd.}\label{\detokenize{index:litilmynd3}}\end{figure}

\begin{figure}[htbp]
\centering
\capstart

\noindent\sphinxincludegraphics[width=400\sphinxpxdimen,height=400\sphinxpxdimen]{{pexels-photo-106399}.jpeg}
\caption{Þetta er caption fyrir stóra mynd.}\label{\detokenize{index:stormynd3}}\end{figure}

\sphinxAtStartPar
Samanber \hyperref[\detokenize{index:stormynd3}]{Mynd \ref{\detokenize{index:stormynd3}}}, þá er þetta mjög flott.


\begin{savenotes}\sphinxattablestart
\sphinxthistablewithglobalstyle
\centering
\sphinxcapstartof{table}
\sphinxthecaptionisattop
\sphinxcaption{Sample Table}\label{\detokenize{index:prufutafla3}}
\sphinxaftertopcaption
\begin{tabulary}{\linewidth}[t]{TTT}
\sphinxtoprule
\sphinxstyletheadfamily 
\sphinxAtStartPar
Header 1
&\sphinxstyletheadfamily 
\sphinxAtStartPar
Header 2
&\sphinxstyletheadfamily 
\sphinxAtStartPar
Header 3
\\
\sphinxmidrule
\sphinxtableatstartofbodyhook
\sphinxAtStartPar
Row 1, Column 1
&
\sphinxAtStartPar
Row 1, Column 2
&
\sphinxAtStartPar
Row 1, Column 3
\\
\sphinxhline
\sphinxAtStartPar
Row 2, Column 1
&
\sphinxAtStartPar
Row 2, Column 2
&
\sphinxAtStartPar
Row 2, Column 3
\\
\sphinxhline
\sphinxAtStartPar
Row 3, Column 1
&
\sphinxAtStartPar
Row 3, Column 2
&
\sphinxAtStartPar
Row 3, Column 3
\\
\sphinxbottomrule
\end{tabulary}
\sphinxtableafterendhook\par
\sphinxattableend\end{savenotes}

\sphinxAtStartPar
Samanber \hyperref[\detokenize{index:prufutafla3}]{Tafla \ref{\detokenize{index:prufutafla3}}}, þá er þetta mjög flott tafla.

\sphinxstepscope


\chapter{Admonitions}
\label{\detokenize{admonitions/index:admonitions}}\label{\detokenize{admonitions/index::doc}}
\sphinxstepscope


\section{Admonitions 1}
\label{\detokenize{admonitions/admonitions-1/index:admonitions-1}}\label{\detokenize{admonitions/admonitions-1/index::doc}}
\sphinxAtStartPar
Text below header one


\subsection{Header two}
\label{\detokenize{admonitions/admonitions-1/index:header-two}}
\sphinxAtStartPar
Text below header two


\subsubsection{Header three}
\label{\detokenize{admonitions/admonitions-1/index:header-three}}
\sphinxAtStartPar
Text below header three


\paragraph{Header four}
\label{\detokenize{admonitions/admonitions-1/index:header-four}}
\sphinxAtStartPar
Text below header four


\subparagraph{Header five}
\label{\detokenize{admonitions/admonitions-1/index:header-five}}
\sphinxAtStartPar
Text below header five


\subparagraph{Header six}
\label{\detokenize{admonitions/admonitions-1/index:header-six}}
\sphinxAtStartPar
Text below header six

\sphinxstepscope


\section{Admonitions 2}
\label{\detokenize{admonitions/admonitions-2/index:admonitions-2}}\label{\detokenize{admonitions/admonitions-2/index::doc}}
\sphinxAtStartPar
Text below header one


\subsection{Header two}
\label{\detokenize{admonitions/admonitions-2/index:header-two}}
\sphinxAtStartPar
Text below header two


\subsubsection{Header three}
\label{\detokenize{admonitions/admonitions-2/index:header-three}}
\sphinxAtStartPar
Text below header three


\paragraph{Header four}
\label{\detokenize{admonitions/admonitions-2/index:header-four}}
\sphinxAtStartPar
Text below header four


\subparagraph{Header five}
\label{\detokenize{admonitions/admonitions-2/index:header-five}}
\sphinxAtStartPar
Text below header five


\subparagraph{Header six}
\label{\detokenize{admonitions/admonitions-2/index:header-six}}
\sphinxAtStartPar
Text below header six

\sphinxstepscope


\section{Admonitions 3}
\label{\detokenize{admonitions/admonitions-3/index:admonitions-3}}\label{\detokenize{admonitions/admonitions-3/index::doc}}
\sphinxAtStartPar
Text below header one


\subsection{Header two}
\label{\detokenize{admonitions/admonitions-3/index:header-two}}
\sphinxAtStartPar
Text below header two


\subsubsection{Header three}
\label{\detokenize{admonitions/admonitions-3/index:header-three}}
\sphinxAtStartPar
Text below header three


\paragraph{Header four}
\label{\detokenize{admonitions/admonitions-3/index:header-four}}
\sphinxAtStartPar
Text below header four


\subparagraph{Header five}
\label{\detokenize{admonitions/admonitions-3/index:header-five}}
\sphinxAtStartPar
Text below header five


\subparagraph{Header six}
\label{\detokenize{admonitions/admonitions-3/index:header-six}}
\sphinxAtStartPar
Text below header six

\sphinxAtStartPar
Hér er smá texti fyrir ofan yfirlit fyrirsafna.

\begin{sphinxShadowBox}
\sphinxstyletopictitle{Yfirlit fyrirsagna í Admonitions}
\begin{itemize}
\item {} 
\sphinxAtStartPar
\phantomsection\label{\detokenize{admonitions/index:id1}}{\hyperref[\detokenize{admonitions/index:header-two}]{\sphinxcrossref{Header two}}}
\begin{itemize}
\item {} 
\sphinxAtStartPar
\phantomsection\label{\detokenize{admonitions/index:id2}}{\hyperref[\detokenize{admonitions/index:header-three}]{\sphinxcrossref{Header three}}}
\begin{itemize}
\item {} 
\sphinxAtStartPar
\phantomsection\label{\detokenize{admonitions/index:id3}}{\hyperref[\detokenize{admonitions/index:header-four}]{\sphinxcrossref{Header four}}}
\begin{itemize}
\item {} 
\sphinxAtStartPar
\phantomsection\label{\detokenize{admonitions/index:id4}}{\hyperref[\detokenize{admonitions/index:header-five}]{\sphinxcrossref{Header five}}}

\end{itemize}

\end{itemize}

\end{itemize}

\end{itemize}
\end{sphinxShadowBox}

\sphinxAtStartPar
Setjum \(6b_y=-6b_x\) inn og fáum:
\begin{equation*}
\begin{split}6b_y = -6b_x\end{split}
\end{equation*}\begin{equation}\label{equation:admonitions/index:euler7}
\begin{split}e^{i\pi} + 1 = 0\end{split}
\end{equation}
\index{Euler\textquotesingle{}s identity@\spxentry{Euler\textquotesingle{}s identity}}\ignorespaces 
\sphinxAtStartPar
Smá annað test:
\begin{equation}\label{equation:admonitions/index:euler8}
\begin{split}e^{i\pi} + 2 = 0\end{split}
\end{equation}
\sphinxAtStartPar
Euler’s identity, equation (\ref{equation:admonitions/index:euler8}), was elected one of the most
beautiful mathematical formulas.

\begin{figure}[htbp]
\centering
\capstart

\noindent\sphinxincludegraphics[width=200\sphinxpxdimen,height=200\sphinxpxdimen]{{pexels-photo-106399}.jpeg}
\caption{Þetta er caption fyrir litla mynd.}\label{\detokenize{admonitions/index:litilmynd4}}\end{figure}

\begin{figure}[htbp]
\centering
\capstart

\noindent\sphinxincludegraphics[width=400\sphinxpxdimen,height=400\sphinxpxdimen]{{pexels-photo-106399}.jpeg}
\caption{Þetta er caption fyrir stóra mynd.}\label{\detokenize{admonitions/index:stormynd4}}\end{figure}

\sphinxAtStartPar
Samanber \hyperref[\detokenize{admonitions/index:stormynd4}]{Mynd \ref{\detokenize{admonitions/index:stormynd4}}}, þá er þetta mjög flott.


\begin{savenotes}\sphinxattablestart
\sphinxthistablewithglobalstyle
\centering
\sphinxcapstartof{table}
\sphinxthecaptionisattop
\sphinxcaption{Sample Table}\label{\detokenize{admonitions/index:prufutafla4}}
\sphinxaftertopcaption
\begin{tabulary}{\linewidth}[t]{TTT}
\sphinxtoprule
\sphinxstyletheadfamily 
\sphinxAtStartPar
Header 1
&\sphinxstyletheadfamily 
\sphinxAtStartPar
Header 2
&\sphinxstyletheadfamily 
\sphinxAtStartPar
Header 3
\\
\sphinxmidrule
\sphinxtableatstartofbodyhook
\sphinxAtStartPar
Row 1, Column 1
&
\sphinxAtStartPar
Row 1, Column 2
&
\sphinxAtStartPar
Row 1, Column 3
\\
\sphinxhline
\sphinxAtStartPar
Row 2, Column 1
&
\sphinxAtStartPar
Row 2, Column 2
&
\sphinxAtStartPar
Row 2, Column 3
\\
\sphinxhline
\sphinxAtStartPar
Row 3, Column 1
&
\sphinxAtStartPar
Row 3, Column 2
&
\sphinxAtStartPar
Row 3, Column 3
\\
\sphinxbottomrule
\end{tabulary}
\sphinxtableafterendhook\par
\sphinxattableend\end{savenotes}

\sphinxAtStartPar
Samanber \hyperref[\detokenize{admonitions/index:prufutafla4}]{Tafla \ref{\detokenize{admonitions/index:prufutafla4}}}, þá er þetta mjög flott tafla.

\sphinxAtStartPar
Text below header one
\sphinxSetupCaptionForVerbatim{This is the code block caption}
\def\sphinxLiteralBlockLabel{\label{\detokenize{admonitions/index:kodabalkur1}}}
\begin{sphinxVerbatim}[commandchars=\\\{\}]
\PYG{k}{def} \PYG{n+nf}{hello\PYGZus{}world}\PYG{p}{(}\PYG{p}{)}\PYG{p}{:}
    \PYG{n+nb}{print}\PYG{p}{(}\PYG{l+s+s2}{\PYGZdq{}}\PYG{l+s+s2}{Hello, world!}\PYG{l+s+s2}{\PYGZdq{}}\PYG{p}{)}
\end{sphinxVerbatim}

\sphinxAtStartPar
Samanber \hyperref[\detokenize{admonitions/index:kodabalkur1}]{Kóðablokk \ref{\detokenize{admonitions/index:kodabalkur1}}}, þá er þetta mjög flottur kóði.


\section{Header two}
\label{\detokenize{admonitions/index:header-two}}
\begin{figure}[htbp]
\centering
\capstart
\caption{Þetta er caption fyrir litla mynd.}\label{\detokenize{admonitions/index:litilmynd5}}\end{figure}

\begin{figure}[htbp]
\centering
\capstart
\caption{Þetta er caption fyrir stóra mynd.}\label{\detokenize{admonitions/index:stormynd5}}\end{figure}

\sphinxAtStartPar
Samanber \hyperref[\detokenize{admonitions/index:stormynd5}]{Mynd \ref{\detokenize{admonitions/index:stormynd5}}}, þá er þetta mjög flott.


\begin{savenotes}\sphinxattablestart
\sphinxthistablewithglobalstyle
\centering
\sphinxcapstartof{table}
\sphinxthecaptionisattop
\sphinxcaption{Sample Table}\label{\detokenize{admonitions/index:prufutafla5}}
\sphinxaftertopcaption
\begin{tabulary}{\linewidth}[t]{TTT}
\sphinxtoprule
\sphinxstyletheadfamily 
\sphinxAtStartPar
Header 1
&\sphinxstyletheadfamily 
\sphinxAtStartPar
Header 2
&\sphinxstyletheadfamily 
\sphinxAtStartPar
Header 3
\\
\sphinxmidrule
\sphinxtableatstartofbodyhook
\sphinxAtStartPar
Row 1, Column 1
&
\sphinxAtStartPar
Row 1, Column 2
&
\sphinxAtStartPar
Row 1, Column 3
\\
\sphinxhline
\sphinxAtStartPar
Row 2, Column 1
&
\sphinxAtStartPar
Row 2, Column 2
&
\sphinxAtStartPar
Row 2, Column 3
\\
\sphinxhline
\sphinxAtStartPar
Row 3, Column 1
&
\sphinxAtStartPar
Row 3, Column 2
&
\sphinxAtStartPar
Row 3, Column 3
\\
\sphinxbottomrule
\end{tabulary}
\sphinxtableafterendhook\par
\sphinxattableend\end{savenotes}

\sphinxAtStartPar
Samanber \hyperref[\detokenize{admonitions/index:prufutafla5}]{Tafla \ref{\detokenize{admonitions/index:prufutafla5}}}, þá er þetta mjög flott tafla.
\sphinxSetupCaptionForVerbatim{This is the code block caption}
\def\sphinxLiteralBlockLabel{\label{\detokenize{admonitions/index:kodabalkur2}}\label{\detokenize{admonitions/index:kodanafn}}}
\begin{sphinxVerbatim}[commandchars=\\\{\}]
\PYG{k}{def} \PYG{n+nf}{hello\PYGZus{}world}\PYG{p}{(}\PYG{p}{)}\PYG{p}{:}
    \PYG{n+nb}{print}\PYG{p}{(}\PYG{l+s+s2}{\PYGZdq{}}\PYG{l+s+s2}{Hello, world!}\PYG{l+s+s2}{\PYGZdq{}}\PYG{p}{)}
\end{sphinxVerbatim}

\sphinxAtStartPar
Samanber \hyperref[\detokenize{admonitions/index:kodabalkur2}]{Kóðablokk \ref{\detokenize{admonitions/index:kodabalkur2}}}, þá er þetta mjög flottur kóði.

\sphinxAtStartPar
Og aftur, samanber {\hyperref[\detokenize{admonitions/index:kodanafn}]{\sphinxcrossref{\DUrole{std,std-ref}{þennan geggjaða kóða}}}}, þá er þetta mjög flottur kóði.


\subsection{Header three}
\label{\detokenize{admonitions/index:header-three}}
\sphinxAtStartPar
Text below header three


\subsubsection{Header four}
\label{\detokenize{admonitions/index:header-four}}
\sphinxAtStartPar
Text below header four


\paragraph{Header five}
\label{\detokenize{admonitions/index:header-five}}
\sphinxAtStartPar
Text below header five


\subparagraph{Header six}
\label{\detokenize{admonitions/index:header-six}}
\sphinxAtStartPar
Text below header six

\sphinxstepscope


\chapter{Blocks}
\label{\detokenize{blocks/index:blocks}}\label{\detokenize{blocks/index::doc}}
\sphinxAtStartPar
Text below header one


\section{Header two}
\label{\detokenize{blocks/index:header-two}}
\sphinxAtStartPar
Text below header two


\subsection{Header three}
\label{\detokenize{blocks/index:header-three}}
\sphinxAtStartPar
Text below header three


\subsubsection{Header four}
\label{\detokenize{blocks/index:header-four}}
\sphinxAtStartPar
Text below header four


\paragraph{Header five}
\label{\detokenize{blocks/index:header-five}}
\sphinxAtStartPar
Text below header five


\subparagraph{Header six}
\label{\detokenize{blocks/index:header-six}}
\sphinxAtStartPar
Text below header six

\sphinxstepscope


\chapter{Generic items}
\label{\detokenize{generic-items/index:generic-items}}\label{\detokenize{generic-items/index::doc}}
\sphinxAtStartPar
Text below header one


\section{Header two}
\label{\detokenize{generic-items/index:header-two}}
\sphinxAtStartPar
Text below header two


\subsection{Header three}
\label{\detokenize{generic-items/index:header-three}}
\sphinxAtStartPar
Text below header three


\subsubsection{Header four}
\label{\detokenize{generic-items/index:header-four}}
\sphinxAtStartPar
Text below header four


\paragraph{Header five}
\label{\detokenize{generic-items/index:header-five}}
\sphinxAtStartPar
Text below header five


\subparagraph{Header six}
\label{\detokenize{generic-items/index:header-six}}
\sphinxAtStartPar
Text below header six

\sphinxstepscope


\chapter{Lists}
\label{\detokenize{lists/index:lists}}\label{\detokenize{lists/index::doc}}
\sphinxAtStartPar
Text below header one


\section{Header two}
\label{\detokenize{lists/index:header-two}}
\sphinxAtStartPar
Text below header two


\subsection{Header three}
\label{\detokenize{lists/index:header-three}}
\sphinxAtStartPar
Text below header three


\subsubsection{Header four}
\label{\detokenize{lists/index:header-four}}
\sphinxAtStartPar
Text below header four


\paragraph{Header five}
\label{\detokenize{lists/index:header-five}}
\sphinxAtStartPar
Text below header five


\subparagraph{Header six}
\label{\detokenize{lists/index:header-six}}
\sphinxAtStartPar
Text below header six

\sphinxstepscope


\chapter{Structural elements}
\label{\detokenize{structural-elements/index:structural-elements}}\label{\detokenize{structural-elements/index::doc}}
\sphinxAtStartPar
Text below header one


\section{Header two}
\label{\detokenize{structural-elements/index:header-two}}
\sphinxAtStartPar
Text below header two


\subsection{Header three}
\label{\detokenize{structural-elements/index:header-three}}
\sphinxAtStartPar
Text below header three


\subsubsection{Header four}
\label{\detokenize{structural-elements/index:header-four}}
\sphinxAtStartPar
Text below header four


\paragraph{Header five}
\label{\detokenize{structural-elements/index:header-five}}
\sphinxAtStartPar
Text below header five


\subparagraph{Header six}
\label{\detokenize{structural-elements/index:header-six}}
\sphinxAtStartPar
Text below header six

\sphinxstepscope


\chapter{Tables}
\label{\detokenize{tables/index:tables}}\label{\detokenize{tables/index::doc}}
\sphinxAtStartPar
Text below header one


\section{Header two}
\label{\detokenize{tables/index:header-two}}
\sphinxAtStartPar
Text below header two


\subsection{Header three}
\label{\detokenize{tables/index:header-three}}
\sphinxAtStartPar
Text below header three


\subsubsection{Header four}
\label{\detokenize{tables/index:header-four}}
\sphinxAtStartPar
Text below header four


\paragraph{Header five}
\label{\detokenize{tables/index:header-five}}
\sphinxAtStartPar
Text below header five


\subparagraph{Header six}
\label{\detokenize{tables/index:header-six}}
\sphinxAtStartPar
Text below header six

\sphinxstepscope


\chapter{Tikz graphics}
\label{\detokenize{tikz-graphics/index:tikz-graphics}}\label{\detokenize{tikz-graphics/index::doc}}
\sphinxAtStartPar
Text below header one


\section{Header two}
\label{\detokenize{tikz-graphics/index:header-two}}
\sphinxAtStartPar
Text below header two


\subsection{Header three}
\label{\detokenize{tikz-graphics/index:header-three}}
\sphinxAtStartPar
Text below header three


\subsubsection{Header four}
\label{\detokenize{tikz-graphics/index:header-four}}
\sphinxAtStartPar
Text below header four


\paragraph{Header five}
\label{\detokenize{tikz-graphics/index:header-five}}
\sphinxAtStartPar
Text below header five


\subparagraph{Header six}
\label{\detokenize{tikz-graphics/index:header-six}}
\sphinxAtStartPar
Text below header six

\sphinxstepscope


\chapter{MathJax renders}
\label{\detokenize{mathjax-renders/index:mathjax-renders}}\label{\detokenize{mathjax-renders/index::doc}}
\sphinxAtStartPar
Text below header one


\section{Header two}
\label{\detokenize{mathjax-renders/index:header-two}}
\sphinxAtStartPar
Text below header two


\subsection{Header three}
\label{\detokenize{mathjax-renders/index:header-three}}
\sphinxAtStartPar
Text below header three


\subsubsection{Header four}
\label{\detokenize{mathjax-renders/index:header-four}}
\sphinxAtStartPar
Text below header four


\paragraph{Header five}
\label{\detokenize{mathjax-renders/index:header-five}}
\sphinxAtStartPar
Text below header five


\subparagraph{Header six}
\label{\detokenize{mathjax-renders/index:header-six}}
\sphinxAtStartPar
Text below header six



\renewcommand{\indexname}{Atriðaskrá}
\printindex
\end{document}